\PassOptionsToPackage{table,svgnames,dvipsnames}{xcolor}
\setcounter{tocdepth}{3}
\setcounter{secnumdepth}{3}
\usepackage[utf8]{inputenc}
\usepackage[T1]{fontenc}
\usepackage[sc]{mathpazo}
\usepackage[ngerman,american]{babel}
\usepackage[autostyle]{csquotes}
\usepackage[%
  backend=biber,
  url=false,
  style=alphabetic,
  maxnames=4,
  minnames=3,
  maxbibnames=99,
  giveninits,
  uniquename=init]{biblatex} % TODO: adapt citation style
\usepackage{graphicx}
\usepackage{scrhack} % necessary for listings package
\usepackage{listings}
\usepackage{lstautogobble}
\usepackage{tikz}
\usetikzlibrary{decorations.text,calc,arrows.meta}
\usetikzlibrary{positioning,fit,arrows.meta,backgrounds}
\usepackage{pgfplots}
\usepackage{pgfplotstable}
\usepackage{booktabs}
\usepackage{soul}
\usepackage[final]{microtype}
\usepackage{caption}
\usepackage[hidelinks]{hyperref} % hidelinks removes colored boxes around references and links
\usepackage{acro}
\DeclareAcronym{IoT}{
    short = IoT ,
    long = Internet of Things,
    tag = abbrev
    }
\DeclareAcronym{JSON}{
    short = JSON ,
    long = JavaScript Object Notation,
    tag = abbrev
    }
\DeclareAcronym{SoC}{
    short = SoC ,
    long = System on Chip,
    tag = abbrev
    }
\DeclareAcronym{BSP}{
    short = BSP ,
    long = Board Support Package,
    tag = abbrev
    }
\DeclareAcronym{CI}{
    short = CI ,
    long = Continuous Integration,
    tag = abbrev
    }
\DeclareAcronym{CD}{
    short = CD ,
    long = Continuous Deployment,
    tag = abbrev
    }
\DeclareAcronym{SDLC}{
    short = SDLC ,
    long = Software Development Life Cycle,
    tag = abbrev
    }
\DeclareAcronym{REST}{
    short = REST ,
    long = REpresentational State Transfer,
    tag = abbrev
    }
\DeclareAcronym{API}{
    short = API ,
    long = Application Programming Interface,
    tag = abbrev
    }
\DeclareAcronym{QA}{
    short = QA ,
    long = Quality Assurance,
    tag = abbrev
    }
\DeclareAcronym{MQTT}{
    short = MQTT ,
    long = Message Queuing Telemetry Transport,
    tag = abbrev
    }
\DeclareAcronym{HTTP}{
    short = HTTP ,
    long = Hypertext Transfer Protocol,
    tag = abbrev
    }
\DeclareAcronym{HTTPS}{
    short = HTTPS ,
    long = Hypertext Transfer Protocol Secure,
    tag = abbrev
    }
\DeclareAcronym{SCM}{
    short = SCM ,
    long = Software Configuration Management,
    tag = abbrev
    }   
\DeclareAcronym{SSTATE}{
    short = SSTATE ,
    long = Software Configuration Management,
    tag = abbrev
    } 
\DeclareAcronym{SVN}{
    short = SVN ,
    long = Subversion,
    tag = abbrev
    }
\DeclareAcronym{FTP}{
    short = FTP ,
    long =  File Transfer Protocol,
    tag = abbrev
    }
\DeclareAcronym{ALM}{
    short = ALM ,
    long = Application Lifecycle Management,
    tag = abbrev
    }
\DeclareAcronym{GUI}{
    short = GUI ,
    long = Graphical User Interface,
    tag = abbrev
    }
\DeclareAcronym{UAT}{
    short = UAT ,
    long = User Acceptance Testing,
    tag = abbrev
    }
\DeclareAcronym{CLI}{
    short = CLI ,
    long = Command Line Interface,
    tag = abbrev
    }
\DeclareAcronym{BDC}{
    short = BDC ,
    long = Bosch Development Cloud,
    tag = abbrev
    }
\DeclareAcronym{OS}{
    short = OS ,
    long = Operating System,
    tag = abbrev
    }
\DeclareAcronym{VCS}{
    short = VCS ,
    long = Version Control System,
    tag = abbrev
    }
\DeclareAcronym{RCS}{
    short = RCS ,
    long = Revision Control System,
    tag = abbrev
    }
\DeclareAcronym{URL}{
    short = URL ,
    long = Uniform Resource Locator,
    tag = abbrev
    }
\DeclareAcronym{DSL}{
    short = DSL ,
    long = Domain Specific Language,
    tag = abbrev
    }
\bibliography{bibliography}

\setkomafont{disposition}{\normalfont\bfseries} % use serif font for headings
\linespread{1.05} % adjust line spread for mathpazo font

% Add table of contents to PDF bookmarks
\BeforeTOCHead[toc]{{\cleardoublepage\pdfbookmark[0]{\contentsname}{toc}}}

% Define TUM corporate design colors
% Taken from http://portal.mytum.de/corporatedesign/index_print/vorlagen/index_farben
\definecolor{TUMBlue}{HTML}{0065BD}
\definecolor{TUMSecondaryBlue}{HTML}{005293}
\definecolor{TUMSecondaryBlue2}{HTML}{003359}
\definecolor{TUMBlack}{HTML}{000000}
\definecolor{TUMWhite}{HTML}{FFFFFF}
\definecolor{TUMDarkGray}{HTML}{333333}
\definecolor{TUMGray}{HTML}{808080}
\definecolor{TUMLightGray}{HTML}{CCCCC6}
\definecolor{TUMAccentGray}{HTML}{DAD7CB}
\definecolor{TUMAccentOrange}{HTML}{E37222}
\definecolor{TUMAccentGreen}{HTML}{A2AD00}
\definecolor{TUMAccentLightBlue}{HTML}{98C6EA}
\definecolor{TUMAccentBlue}{HTML}{64A0C8}

% Settings for pgfplots
\pgfplotsset{compat=newest}
\pgfplotsset{
  % For available color names, see http://www.latextemplates.com/svgnames-colors
  cycle list={TUMBlue\\TUMAccentOrange\\TUMAccentGreen\\TUMSecondaryBlue2\\TUMDarkGray\\},
}

% Settings for lstlistings
\lstset{%
  basicstyle=\ttfamily,
  columns=fullflexible,
  autogobble,
  keywordstyle=\bfseries\color{TUMBlue},
  stringstyle=\color{TUMAccentGreen}
}

\usepackage{xcolor}
\usepackage[dvipsnames]{xcolor}
\definecolor{codegreen}{RGB}{2,112,10}
\definecolor{codegray}{rgb}{0.5,0.5,0.5}
\definecolor{eminence}{RGB}{108,48,130}
%\definecolor{backcolour}{rgb}{0.95,0.95,0.92}
\definecolor{weborange}{RGB}{255,165,0}

\lstdefinestyle{mystyle}{
%    backgroundcolor=\color{backcolour},   
    commentstyle=\color{codegreen},
    keywordstyle=\color{weborange},
    numberstyle=\tiny\color{codegray},
    stringstyle=\color{eminence},
    basicstyle=\ttfamily\footnotesize,
    breakatwhitespace=false,         
    breaklines=true,                 
    captionpos=b,                    
    keepspaces=true,                 
    numbers=left,                    
    numbersep=5pt,                  
    showspaces=false,                
    showstringspaces=false,
    showtabs=false,                  
    tabsize=2
}

\tikzset{
    module/.style={%
        draw, rounded corners,
        minimum width=#1,
        minimum height=7mm,
        font=\sffamily
        },
    module/.default=2cm,
    >=LaTeX
}
\usepackage{floatrow}
\usepackage{blindtext}
\usepackage[edges]{forest}
\definecolor{folderbg}{RGB}{124,166,198}
\definecolor{folderborder}{RGB}{110,144,169}
\newlength\Size
\setlength\Size{4pt}
\tikzset{%
  folder/.pic={%
    \filldraw [draw=folderborder, top color=folderbg!50, bottom color=folderbg] (-1.05*\Size,0.2\Size+5pt) rectangle ++(.75*\Size,-0.2\Size-5pt);
    \filldraw [draw=folderborder, top color=folderbg!50, bottom color=folderbg] (-1.15*\Size,-\Size) rectangle (1.15*\Size,\Size);
  },
  file/.pic={%
    \filldraw [draw=folderborder, top color=folderbg!5, bottom color=folderbg!10] (-\Size,.4*\Size+5pt) coordinate (a) |- (\Size,-1.2*\Size) coordinate (b) -- ++(0,1.6*\Size) coordinate (c) -- ++(-5pt,5pt) coordinate (d) -- cycle (d) |- (c) ;
  },
}
\forestset{%
  declare autowrapped toks={pic me}{},
  pic dir tree/.style={%
    for tree={%
      folder,
      font=\ttfamily,
      grow'=0,
    },
    before typesetting nodes={%
      for tree={%
        edge label+/.option={pic me},
      },
    },
  },
  pic me set/.code n args=2{%
    \forestset{%
      #1/.style={%
        inner xsep=2\Size,
        pic me={pic {#2}},
      }
    }
  },
  pic me set={directory}{folder},
  pic me set={file}{file},
}
\usepackage{enumitem}
\renewbibmacro{in:}{}
